\setlength\parindent{0pt}
\chapter{Lista de símbolos}

$U$: Potencial gravitacional escalar.

$\rho$: densidade.

$x, y, z$ Sistema de coordenadas Cartesianas principal.

$(x_{c}, y_{c}, z_{c})$: Centro do elipsoide.

$a, b, c$: Semi-eixos maior, intermediário e menor, respectivamente.

$\mathbf{r}$: vetor que define um ponto na superfície do elipsoide.

$\mathbf{r}_c$: vetor que define o centro do elipsoide.

$\varepsilon$, $\zeta$, $\eta$: Orientações do elipsoide (\textit{strike}, \textit{dip}, \textit{rake}, respectivamente), utilizadas para calcular $\alpha$, $\gamma$, e $\delta$.

$\mathbf{V}$: Matriz de transformação de coordenadas, cujas colunas são definidas por vetores unitários $\mathbf{v}_{1}$, $\mathbf{v}_{2}$, e $\mathbf{v}_{3}$, calculadas a partir dos ângulos $\alpha$, $\gamma$, e $\delta$.

$\tilde{ }\,\,$: Símbolo que denota grandezas referidas ao sistema de coordenadas local.

$\mathbf{{H}_{0}}$: Campo magnético local (intensidade).

$\mathbf{H}$: Campo magnético total.

$V$: Potencial magnético escalar.

$\mathbf{M}$: Vetor de magnetização.

${\dagger}$: Símbolo que denota grandezas referidas dentro do corpo.

${\ddagger}$: Símbolo que denota grandezas referidas fora do corpo.

$\mathbf{K}$: Tensor de segunda ordem, que representa a susceptibilidade magnética do corpo.

$k_1,k_2, k_3$: valores de susceptibilidade principais.

$\mathbf{U}$: Matriz que contém os vetores unitários das direções de susceptibilidade.

$\chi$: Susceptibilidade para o caso isotrópico.

$\mathbf{N}$: Tensor de segunda ordem, que representa a depolarização.

$n^{\dagger}_{11}$, $n^{\dagger}_{22}$, $n^{\dagger}_{33}$: Fatores de desmagnetização, referentes ao semi-eixos maior, intermediário e menor, respectivamente.

$\lambda$: maior raiz real da equação cúbica que define a superfície de um elipsoide (ver Apêndice B).

$F(\kappa, \phi)$, $E(\kappa, \phi)$: Integrais elípticas normais de Legendre de primeiro e segundo tipo, respectivamente.

$\epsilon$: Erro relativo máximo da magnetização.

$\Delta {\mathbf{H}}$: O campo magnético produzido por um elipsoide.

$\Delta {T}$: Anomalia de campo total.

${\mathbf{B}}_{0}$: Campo  magnético indutor local.

$\Delta \mathbf{B}$: Campo magnético induzido pelo elipsoide.