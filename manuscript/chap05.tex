\chapter{Conclusões}

Esta dissertação apresenta uma revisão da vasta literatura sobre a modelagem magnética de elipsoides triaxiais, prolatos e oblatos, com orientação espacial arbitrária, suscetibilidade anisotrópica e magnetização remanente. Conjuntamente, apresenta uma forma alternativa de determinar a suscetibilidade isotrópica a partir da qual a desmagnetização deve ser considerada na modelagem e também a implementação de uma série de rotinas para o cálculo do campo magnético de corpos elipsoidais.

A modelagem direta do campo magnético gerado por fontes elipsoidais se mostrou satisfatória para os testes conduzidos neste trabalho. As componentes do campo magnético e a anomalia de campo total foram calculadas conforme o esperado para as três implementações propostas: para elipsoides triaxiais, prolatos e oblatos.
As validações das rotinas desenvolvidas neste trabalho foram feitas comparando-se as anomalias de campo total produzida por corpos com formato similar e localizados na mesma posição. As comparações foram feitas utilizando-se: i) um elipsoide triaxial e uma esfera, ii) um elipsoide triaxial e um elipsoide prolato e iii) um elipsoide triaxial e um elipsoide oblato. Os dados produzidos pelos elipsoides foram calculados com as rotinas desenvolvidas neste trabalho. Já os dados produzidos pela esfera foram calculados com o pacote Fatiando a Terra. Em todos os casos, as anomalias calculadas ficaram muito próximas entre si.

Outra simulação mostrou que os fatores de desmagnetização calculados para todos os tipos de elipsoide são i) diferentes  uns dos outros quando os semi-eixos dos elipsoides têm valores diferentes entre si e ii) tendem a $1/3$ quando o tamanho dos semi-eixos dos elipsoides são próximos uns aos outros.

Testes numéricos mostraram a mudança da direção do vetor de magnetização resultante em função da intensidade da susceptibilidade isotrópica. Estes testes ilustraram o critério alternativo proposto nesta dissertação para determinar o valor de suscetibilidade isotrópica a partir do qual a desmagnetização deve ser levada em consideração na modelagem magnética de corpos elipsoidais. Este critério é baseado no conhecimento prévio sobre a forma do corpo e no máximo erro relativo permitido na magnetização. Este critério é uma alternativa ao valor de 0,1 SI que tem sido utilizado de forma empírica pela comunidade geocientífica.

As rotinas desenvolvidas aqui se mostraram eficazes para calcular, de forma relativamente simples, o campo magnético produzido por múltiplos corpos. Estas rotinas foram desenvolvidas com base no pacote \textit{Fatiando a Terra} com o intuito de torná-las acessíveis para a comunidade científica. As rotinas estão disponibilizadas livremente no seguinte repositório no GitHub: https://github.com/DiegoTaka/ellipsoid-magnetic.

Futuramente estas rotinas poderão ser utilizadas tanto para modelar situações geológicas reais quanto para ensino. Estudos teóricos também poderão ser feitos para determinar o valor de suscetibilidade isotrópica a partir do qual a desmagnetização pode ser negligenciada com base no erro relativo máximo permitido no campo magnético calculado. As rotinas também poderão ser utilizadas para estudos de inversão magnética, podendo-se estimar diversos parâmetros como o vetor de magnetização, semi-eixos e as orientações angulares do corpo elipsoidal.

