\chapter{Conclusões}

Apresentamos neste trabalho uma revisão da vasta literatura sobre a modelagem magnética de elipsoides triaxiais, prolatos e oblatos. Conjuntamente, realizamos a implementação de uma série de rotinas para o cálculo do campo magnético destes corpos.

A modelagem direta do campo magnético gerado por fontes elipsoidais se mostrou satisfatória para os testes conduzidos neste trabalho. As componentes do campo magnético e a anomalia de campo total foram calculadas conforme o esperado para as três implementações propostas: para elipsoides triaxiais, prolatos e oblatos.
Para validação do nosso código realizamos algumas simulações. Foi comparada a modelagem do elipsoide triaxial com uma esfera, do triaxial com um elipsoide prolato e por último do triaxial com um elipsoide oblato. Em todos os casos obtivemos resultados positivos.

Os elementos dos fatores de desmagnetização se comportaram conforme o esperado para todos os tipos de elipsoide, afastados uns dos outros quando os semi-eixos tem valores diferentes e todos tendendo a $1/3$ quando próximos.

Para o teste de susceptibilidade, foi calculado a mudança da direção do vetor de magnetização resultante, conforme a intensidade da susceptibilidade isotrópica aumenta. Apresentamos uma discussão teórica sobre a determinação do valor para o qual a desmagnetização deve ser levada em consideração. Propusemos um meio de determinar este valor baseado apenas na forma do corpo e o máximo erro relativo permitido na magnetização. O valor de 0,1 SI da susceptibilidade tem sido utilizado de forma empírica pela comunidade geocientífica e apresentamos aqui uma abordagem alternativa para seu cálculo.

A implementação também se mostrou eficaz para modelar, de forma prática, múltiplos corpos de uma única vez, utilizando a classe elipsoidal. Estas rotinas farão parte do pacote \textit{Fatiando a Terra}, e por isso, é também beneficiado por ter à disposição as ferramentas já nele implementadas, junto com sua estrutura que permite acessibilidade e maior difusão na comunidade científica. Através do repositório no GitHub (https://github.com/DiegoTaka/ellipsoid-magnetic) é possível ter acesso livre ao código.

Futuramente este código poderá ser utilizado para modelar situações mais complexas e também situações geológicas reais. Devido a particularidade do corpo elipsoidal em ter solução analítica para a desmagnetização, estudos mais profundos sobre a susceptibilidade magnética e a relação desta com o erro da modelagem magnética poderão ser feitos. O código também poderá servir para estudos de inversão magnética, podendo-se estimar diversos parâmetros como o vetor de magnetização, semi-eixos e as orientações angulares do corpo elipsoidal.

