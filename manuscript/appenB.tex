\chapter{Parâmetro $\lambda$ e suas derivadas espaciais}

Aqui, mostramos o raciocínio apresentado por \citet{webster1904}
para análise do parâmetro $\lambda$ que define os elipsoides triaxiais, prolatos e oblatos.

\section{Parâmetro $\lambda$ definido para os elipsoides triaxiais}

Consideremos um elipsoide de semi-eixos $a$, $b$, $c$ orientados ao longo dos eixos
$\tilde{x}$, $\tilde{y}$, e $\tilde{z}$, respectivamente, no seu sistema de coordenadas
local, onde $a > b > c > 0$. Este elipsoide é definido pela seguinte equação:
\begin{equation}
\frac{\tilde{x}^{2}}{a^{2}} + \frac{\tilde{y}^{2}}{b^{2}} + \frac{\tilde{z}^{2}}{c^{2}} = 1 \: .
\label{eq:reference-triaxial-ell}
\end{equation}
Uma superfície quádrica (e.g., elipsoide, hiperboloides de uma camada ou 
hiperboloides de duas camadas) que é confocal à um elipsoide definido pela
Eq. \ref{eq:reference-triaxial-ell} pode ser descrito como:
\begin{equation}
\frac{\tilde{x}^{2}}{a^{2} + u} + \frac{\tilde{y}^{2}}{b^{2} + u} + \frac{\tilde{z}^{2}}{c^{2} + u} = 1 \: ,
\label{eq:quadric-confocal-triaxial-ell}
\end{equation}
onde $u$ é um número real. A equação \ref{eq:quadric-confocal-triaxial-ell}
representa um elipsoide para $u$ que satisfaz a condição:
\begin{equation}
u + c^{2} > 0 \: .
\label{eq:condition-triaxial-ell}
\end{equation}

Dado $a$, $b$, $c$, e um $u$ que satisfaz \ref{eq:condition-triaxial-ell}, 
podemos usar \ref{eq:quadric-confocal-triaxial-ell} para determinar um conjunto 
de pontos $(x, y, z)$ situado na superfície de um elipsoide que é confocal aquele
definido pela Eq. \ref{eq:reference-triaxial-ell}. 
Considere o problema em determinar o elipsoide que é confocal com aquele definido em
\ref{eq:reference-triaxial-ell} e que passa por um ponto particular $(\tilde{x}, \tilde{y}, \tilde{z})$.
Este problema consiste em determinar o número real $u$ que, dado  $a$, $b$, $c$, $\tilde{x}$, $\tilde{y}$, e $\tilde{z}$,
satisfaz a Eq. \ref{eq:quadric-confocal-triaxial-ell} e
a condição expressa pela Eq. \ref{eq:condition-triaxial-ell}.
Rearranjando a Eq. \ref{eq:quadric-confocal-triaxial-ell}, obtemos a seguinte equação cúbica para $u$
\begin{equation}
p(u) = (a^{2} + u)(b^{2} + u)(c^{2} + u) - (b^{2} + u)(c^{2} + u) \, \tilde{x}^{2}
- (a^{2} + u)(c^{2} + u) \, \tilde{y}^{2} - (a^{2} + u)(b^{2} + u) \, \tilde{z}^{2} \: .
\label{eq:cubic-equation-triaxial-ell}
\end{equation}
Esta equação cúbica mostra que:
\begin{equation}
u = \begin{cases}
d \to \infty \: &, \quad p(u) > 0 \\
-c^{2} \: &, \quad p(u) < 0 \\
-b^{2} \: &, \quad p(u) > 0 \\
-a^{2} \: &, \quad p(u) < 0
\end{cases} \: .
\label{eq:cubic-equation-signals-triaxial-ell}
\end{equation}
Note que, de acordo com \ref{eq:cubic-equation-signals-triaxial-ell},
a menor, a intermediária e a maior raiz da equação cúbica
$p(u)$ (Eq. \ref{eq:cubic-equation-triaxial-ell}) estão 
localizadas, respectivamente, nos intervalos $[ -a^{2} \, , -b^{2} ]$,
$[ -b^{2} \, , -c^{2} ]$ and $[ -c^{2} \, , \infty [$.
Lembremos que estamos procurando um $u$ que satisfaz a condição
expressa pela Eq. \ref{eq:condition-triaxial-ell}. 
Consequentemente, de acordo com a análise de sinal mostrada na Eq.
\ref{eq:cubic-equation-signals-triaxial-ell}, estamos interessados na 
maior raiz $\lambda$ da equação cúbica $p(u)$ (Eq.
\ref{eq:cubic-equation-triaxial-ell}).

Da Eq. \ref{eq:cubic-equation-triaxial-ell}, obtemos uma mais simples
dada por:
\begin{equation}
p(u) =  u^{u} + p_{2} \, u^{2} + p_{1} \, u + p_{0} \: ,
\label{eq:cubic-equation-triaxial-ell-simpler}
\end{equation}
onde
\begin{equation}
p_{2} = a^{2} + b^{2} + c^{2} - \tilde{x}^{2} - \tilde{y}^{2} - \tilde{z}^{2} \: ,
\label{eq:p2-triaxial-ell}
\end{equation}
\begin{equation}
p_{1} = b^{2} \, c^{2} + a^{2} \, c^{2} + a^{2} \, b^{2} 
- (b^{2} + c^{2}) \, \tilde{x}^{2}
- (a^{2} + c^{2}) \, \tilde{y}^{2} 
- (a^{2} + b^{2}) \, \tilde{z}^{2}
\label{eq:p1-triaxial-ell}
\end{equation}
e
\begin{equation}
p_{0} =  a^{2} \, b^{2} \, c^{2} - b^{2} \, c^{2} \, 
\tilde{x}^{2} - a^{2} \, c^{2} \, \tilde{y}^{2} - a^{2} \, 
b^{2} \, \tilde{z}^{2} \: .
\label{eq:p0-triaxial-ell}
\end{equation}
Finalmente, das Eqs. \ref{eq:p2-triaxial-ell}, 
\ref{eq:p1-triaxial-ell} e \ref{eq:p0-triaxial-ell},
a maior raiz $\lambda$ de $p(u)$ 
(Eq. \ref{eq:cubic-equation-triaxial-ell-simpler}) pode ser
calculada como mostra \citep{weisstein2017}:
\begin{equation}
\lambda = 2 \, \sqrt{-Q} \, \cos \left( \frac{\theta}{3}\right) - \frac{p_{2}}{3} \: ,
\label{eq:lambda-triaxial-ell}
\end{equation}
onde
\begin{equation}
\theta = \cos^{-1} \left( \frac{R}{\sqrt{Q^{3}}} \right) \: ,
\label{eq:theta-triaxial-ell}
\end{equation}
\begin{equation}
Q = \frac{3 \, p_{1} - p_{2}^{2}}{9}
\label{eq:Q-triaxial-ell}
\end{equation}
e
\begin{equation}
R = \frac{9 \, p_{1} \, p_{2} - 27 \, p_{0} - 2 \, p_{2}^{3}}{54} \: .
\label{eq:R-triaxial-ell}
\end{equation}

\section{Parâmetro $\lambda$ que define os elipsoides prolatos e oblatos}

Vamos agora considerar um elipsoide prolato de semi-eixos $a$, $b$, 
$c$ orientados ao longo dos eixos $\tilde{x}$, $\tilde{y}$, e 
$\tilde{z}$, respectivamente, de seu sistema de coordenadas local, onde $a > b = c > 0$.
Neste caso, a equação que define a superfície do elipsoide é obtida substituindo
$c = b$ na Eq. \ref{eq:reference-triaxial-ell}. Consequentemente, a equação que define a
respectiva superfície quádrica confocal é dada por:
\begin{equation}
\frac{\tilde{x}^{2}}{a^{2} + u} + \frac{\tilde{y}^{2} + \tilde{z}^{2}}{b^{2} + u} = 1
\label{eq:quadric-confocal-prolate-ell}
\end{equation}
e a nova condição que a variável $u$ precisa satisfazer para que a
Eq. \ref{eq:quadric-confocal-prolate-ell} represente um elipsoide é:
\begin{equation}
u + b^{2} > 0 \: .
\label{eq:condition-prolate-ell}
\end{equation}

Similarmente ao caso do elipsoide triaxial apresentado na subseção anterior,
estamos interessados em determinar o número real $u$ que, dado $a$, $b$, $\tilde{x}$, $\tilde{y}$,
e $\tilde{z}$, satisfaz a Eq. \ref{eq:quadric-confocal-prolate-ell} e a condição expressa por
Eq. \ref{eq:condition-prolate-ell}. Da Eq. \ref{eq:quadric-confocal-prolate-ell}, obtemos a
seguinte equação quádrica para $u$:
\begin{equation}
p(u) = (a^{2} + u)(b^{2} + u) - (b^{2} + u) \, \tilde{x}^{2}
- (a^{2} + u) \, (\tilde{y}^{2} + \tilde{z}^{2}) \: .
\label{eq:quadratic-equation-prolate-ell}
\end{equation}
Esta equação mostra que:
\begin{equation}
u = \begin{cases}
d \to \infty \: &, \quad f(\rho) > 0 \\
-b^{2} \: &, \quad f(\rho) < 0 \\
-a^{2} \: &, \quad f(\rho) > 0
\end{cases}
\label{eq:quadratic-equation-signals-prolate-ell}
\end{equation}
e, consequentemente, que duas raízes se situam nos intervalos $[ -a^{2} \, , -b^{2} ]$ 
e $[ -b^{2} \, , \infty [$. Assim, de acordo com a condição estabelecida pela
Eq. \ref{eq:condition-prolate-ell} e a análise de sinais mostrada na Eq. \ref{eq:quadratic-equation-signals-prolate-ell},
estamos interessados na maior raiz $\lambda$ da equação quádrica $p(u)$ (Eq. \ref{eq:quadratic-equation-prolate-ell}).

Manipulando corretamente a Eq. \ref{eq:quadratic-equation-prolate-ell}, obtemos uma mais simples dada por:
\begin{equation}
p(u) =  u^{2} + p_{1} \, u + p_{0} \: ,
\label{eq:quadratic-equation-prolate-ell-simpler}
\end{equation}
onde
\begin{equation}
p_{1} = a^{2} + b^{2} - \tilde{x}^{2} - \tilde{y}^{2} - \tilde{z}^{2}
\label{eq:p1-prolate-ell}
\end{equation}
e
\begin{equation}
p_{0} =  a^{2} \, b^{2} 
- b^{2} \, \tilde{x}^{2} 
- a^{2} \left( \tilde{y}^{2} + \tilde{z}^{2} \right) \: .
\label{eq:p0-prolate-ell}
\end{equation}
Finalmente, usando as Eqs. \ref{eq:p1-prolate-ell} e
\ref{eq:p0-prolate-ell}, a maior raiz $\lambda$ de $p(u)$ 
(Eq. \ref{eq:quadratic-equation-prolate-ell-simpler}) pode ser 
facilmente calculada como:
\begin{equation}
\lambda = \frac{-p_{1} + \sqrt{p_{1}^{2} - 4 \, p_{0}}}{2} \: .
\label{eq:lambda-prolate-ell}
\end{equation}

Para o caso do elipsoide oblato, o procedimento para determinar o parâmetro
$\lambda$ é muito similar para os elipsoides prolatos. Os semi-eixos $a$, $b$, $c$
do elipsoide oblato são definidos de forma que $b = c > a > 0$ e a condição que a variável $u$
deve satisfazer é $u + a^{2} > 0$. Neste caso, as duas raízes que resultam da equação quádrica situam-se
nos intervalos $[ -b^{2} \, , -a^{2} ]$ e $[ -a^{2} \, , \infty [$. Consequentemente, continuamos interessados 
na maior raiz da equação quádrica para a variável $u$, que também é calculada usando a Eq. \ref{eq:lambda-prolate-ell}.

\section{Derivadas espaciais do parâmetro $\lambda$}

A modelagem magnética dos elipsoides triaxiais, prolatos e oblatos necessitam além do parâmetro
$\lambda$ definidos pelas Eqs. \ref{eq:lambda-triaxial-ell} e \ref{eq:lambda-prolate-ell}, das suas derivadas
com respeito as coordenadas espaciais $\tilde{x}$, $\tilde{y}$, and $\tilde{z}$.
Felizmente, as derivadas espaciais do parâmetro $\lambda$ podem ser calculadas de forma bastante
similares para todos os tipos de elipsoides.

Vamos considerar um elipsoide triaxial primeiro. Neste caso, as derivas espaciais de $\lambda$
são dadas por:
\begin{equation}
\frac{\partial \lambda}{\partial \tilde{r}_{j}} =
\frac{\frac{2 \, \tilde{r}_{j}}{\left( e_{j}^{2} + \lambda \right)}}{
	\left( \frac{\tilde{x}}{a^{2} + \lambda}\right)^{2} +
	\left( \frac{\tilde{y}}{b^{2} + \lambda}\right)^{2} + 
	\left( \frac{\tilde{z}}{c^{2} + \lambda}\right)^{2}} \: , \quad j = 1, 2, 3 \: ,
\label{eq:dlambda}
\end{equation}
onde 
$\tilde{r}_{1} = \tilde{x}$, $\tilde{r}_{2} = \tilde{y}$, $\tilde{r}_{3} = \tilde{z}$,
$e_{1} = a$, $e_{2} = b$, and $e_{3} = c$.
Esta equação pode ser determinada diretamente da equação \ref{eq:quadric-confocal-triaxial-ell}.
As derivadas espaciais de $\lambda$ no caso dos elipsoides prolatos e oblatos podem ser calculados usando
a Eq. \ref{eq:dlambda} para o o caso particular em que $b = c$.
