\chapter{Simulações Numéricas e Discussões}

\section{Modelos elipsoidais}

As Figuras \ref{fig:triaxial}A, \ref{fig:triaxial}B, \ref{fig:triaxial}C e \ref{fig:triaxial}D mostram, respectivamente, as componentes $x$, $y$ e $z$ da indução magnética $\Delta \mathbf{B}(\mathbf{r})$ (Eq. \ref{eq:delta-T-tilde}) e a
anomalia de campo total $\Delta T (\mathbf{r})$ (Eq. \ref{eq:delta-T-tilde-approx}) produzidas pelo elipsoide triaxial definido de acordo com os parâmetros da Tabela \ref{tab:triaxial}.
Os dados foram calculados em uma malha de $200 \times 200$ pontos (total de 40000 pontos) localizada no plano horizontal $z = 0$ m.

\begin{table}[h]
	\begin{center}
		\begin{tabular}{|l|c|c|}
			\hline
			\textbf{Parâmetro}  & \textbf{Valor}  & \textbf{Unidade} \\
			\hline 
			$a$, $b$, $c$   & 150, 100, 75 & m, m, m\\
			\hline
			\textit{strike}   & $0$ & $^{\circ}$\\
			\hline
			\textit{dip} & $0$ & $^{\circ}$\\
			\hline
			\textit{rake} & $0$  & $^{\circ}$\\
			\hline
			$x_{c}$   & 0  & m\\
			\hline          
			$y_{c}$   & 0  & m\\
			\hline                
			$z_{c}$   & 1000 & m\\
			\hline
			$\mathbf{M}_{R}$*  & 100, $25$, $40$  & $A/m$, $^{\circ}$, $^{\circ}$\\
			\hline
			$\mathbf{B}_{0}$*    & 60000, $50$, $20$ & $nT$, $^{\circ}$, $^{\circ}$\\
			\hline
			$k_{1}$, $k_{2}$, $k_{3}$   & 0,01, 0,01, 0,01  & SI, SI, SI\\
			\hline
			Orientação das colunas da matriz $\mathbf{U}$**   & $0$, $90$, $90$  & $^{\circ}$, $^{\circ}$, $^{\circ}$\\
			\hline
		\end{tabular}
		\caption{Parâmetros do elipsoide triaxial que produz o campo magnético mostrado na Figura \ref{fig:triaxial}. *Valores de intensidade, inclinação e declinação respectivamente. **Ângulos de \textit{strike}, \textit{dip}  e \textit{rake}, respectivamente, utilizados para calcular os vetores unitários $\mathbf{u}_{1}$, $\mathbf{u}_{2}$, $\mathbf{u}_{3}$ por meio das Eqs. \ref{eq:v1_triaxial_prolate}, \ref{eq:v2_triaxial_prolate} e \ref{eq:v3_triaxial_prolate}.}
	\end{center}
	\label{tab:triaxial}
\end{table}
\newpage

\begin{figure}[hbt!]
	\centering \includegraphics[width=16cm,height=16cm]{figures/ellipsoid_triaxial}
	\caption[Campo gerado pelo elipsoide triaxial definido na Tabela \ref{tab:triaxial}. Componentes A) $x$, B) $y$ e C) $z$ da indução magnética $\Delta \mathbf{B}(\mathbf{r})$ (Eq. \ref{eq:delta-T-tilde}). D) Anomalia de campo total $\Delta T (\mathbf{r})$ (Eq. \ref{eq:delta-T-tilde-approx}). Todos os valores estão em nT.]{Campo gerado pelo elipsoide triaxial definido na Tabela \ref{tab:triaxial}. Componentes A) $x$, B) $y$ e C) $z$ da indução magnética $\Delta \mathbf{B}(\mathbf{r})$ (Eq. \ref{eq:delta-T-tilde}). D) Anomalia de campo total $\Delta T (\mathbf{r})$ (Eq. \ref{eq:delta-T-tilde-approx}). Todos os valores estão em nT.}
	\label{fig:triaxial}
\end{figure}

As Figuras \ref{fig:prolate}A), \ref{fig:prolate}B), \ref{fig:prolate}C) e \ref{fig:prolate}D) mostram, respectivamente, as componentes $x$, $y$ e $z$ da indução magnética $\Delta \mathbf{B}(\mathbf{r})$ (Eq. \ref{eq:delta-T-tilde}) e a
anomalia de campo total $\Delta T (\mathbf{r})$ (Eq. \ref{eq:delta-T-tilde-approx}) produzidas pelo elipsoide prolato definido de acordo com os parâmetros da Tabela 4.2.

As Figuras \ref{fig:oblate}A), \ref{fig:oblate}B), \ref{fig:oblate}C) e \ref{fig:oblate}D) mostram, respectivamente, as componentes $x$, $y$ e $z$ da indução magnética $\Delta \mathbf{B}(\mathbf{r})$ (Eq. \ref{eq:delta-T-tilde}) e a
anomalia de campo total $\Delta T (\mathbf{r})$ (Eq. \ref{eq:delta-T-tilde-approx}) produzidas pelo elipsoide prolato definido de acordo com os parâmetros da Tabela 4.3.
\\\\\\\\

\begin{table}[h!]
	\begin{center}
		\begin{tabular}{|l|c|c|}
			\hline
			\textbf{Parâmetro}  & \textbf{Valor}  & \textbf{Unidade} \\
			\hline 
			$a$, $b$   & 200, 100 & m, m, m\\
			\hline
			\textit{strike}   & $45$ & $^{\circ}$\\
			\hline
			\textit{dip}    & $0$ & $^{\circ}$\\
			\hline
			\textit{rake}   & $0$  & $^{\circ}$\\
			\hline
			$x_c$   & 0  & m\\
			\hline          
			$y_c$   & 0  & m\\
			\hline                
			$z_c$   & 1000  & m\\
			\hline
			$\mathbf{M}_{R}$*  & 100, $90$, $0$ & $A/m$, $^{\circ}$, $^{\circ}$\\
			\hline
			$\mathbf{B}_{0}$*    & 60000, $50$, $20$ & $nT$, $^{\circ}$, $^{\circ}$\\
			\hline
			$k_{1}$, $k_{2}$, $k_{3}$   & 0,01, 0,01, 0,01  & SI, SI, SI\\
			\hline
			Orientação das colunas da matriz $\mathbf{U}$**   & $0$, $90$, $90$  & $^{\circ}$, $^{\circ}$, $^{\circ}$\\
			\hline
			
		\end{tabular}
		\caption{Parâmetros do elipsoide prolato que produz o campo magnético mostrado na Figura \ref{fig:prolate}. *Valores de intensidade, inclinação e declinação respectivamente. **Ângulos de \textit{strike}, \textit{dip}  e \textit{rake}, respectivamente, utilizados para calcular os vetores unitários $\mathbf{u}_{1}$, $\mathbf{u}_{2}$, $\mathbf{u}_{3}$ por meio das Eqs. \ref{eq:v1_triaxial_prolate}, \ref{eq:v2_triaxial_prolate} e \ref{eq:v3_triaxial_prolate}.}
	\end{center}
	\label{tab:prolato}
\end{table}

\begin{table}[h!]
	\begin{center}
		\begin{tabular}{lc}
		
			 &  \\
			 & \\
			 & \\
			 & \\
\end{tabular}
\end{center}
\end{table}

\begin{table}[h!]
	\begin{center}
		\begin{tabular}{|l|c|c|}
			\hline
			\textbf{Parâmetro}  & \textbf{Valor} & \textbf{Unidade} \\
			\hline 
			$a$, $b$  & 100, 200 & m, m, m \\
			\hline
			\textit{strike}   & $45$ &  $^{\circ}$\\
			\hline
			\textit{dip}    & $0$ &  $^{\circ}$\\
			\hline
			\textit{rake}   & $0$  &  $^{\circ}$\\
			\hline
			$x_c $  & 0  & m\\
			\hline          
			$y_c$   & 0  & m\\
			\hline                
			$z_c$   & 1000  & m\\
			\hline
			$\mathbf{M}_{R}$*  & 100, $90$, $0$  & $A/m$, $^{\circ}$, $^{\circ}$\\
			\hline
			$\mathbf{B}_{0}$*    & 60000, $50$, $20$ & $nT$, $^{\circ}$, $^{\circ}$\\
			\hline
			$k_{1}$, $k_{2}$, $k_{3}$   & 0,01, 0,01, 0,01  & SI, SI, SI\\
			\hline
			Orientação das colunas da matriz $\mathbf{U}$**   & $0$, $90$, $90$  &  $^{\circ}$,  $^{\circ}$,  $^{\circ}$\\
			\hline
		\end{tabular}
		\caption{Parâmetros do elipsoide oblato que produz o campo magnético mostrado na Figura \ref{fig:oblate}. *Valores de intensidade, inclinação e declinação respectivamente. **Ângulos de \textit{strike}, \textit{dip}  e \textit{rake}, respectivamente, utilizados para calcular os vetores unitários $\mathbf{u}_{1}$, $\mathbf{u}_{2}$, $\mathbf{u}_{3}$ por meio das Eqs. \ref{eq:v1_oblate}, \ref{eq:v2_oblate} e \ref{eq:v3_oblate}.}
	\end{center}
	\label{tab:oblate}
\end{table}

\begin{figure}[hbt!]
	\centering \includegraphics[width=16cm,height=16cm]{figures/ellipsoid_prolate}
	\caption[Campo gerado pelo elipsoide prolato definido na Tabela 4.2. Componentes A) $x$, B) $y$ e C) $z$ da indução magnética $\Delta \mathbf{B}(\mathbf{r})$ (Eq. \ref{eq:delta-T-tilde}). D) Anomalia de campo total $\Delta T (\mathbf{r})$ (Eq. \ref{eq:delta-T-tilde-approx}). Todos os valores estão em nT.]{Campo gerado pelo elipsoide prolato definido na Tabela 4.2. Componentes A) $x$, B) $y$ e C) $z$ da indução magnética $\Delta \mathbf{B}(\mathbf{r})$ (Eq. \ref{eq:delta-T-tilde}). D) Anomalia de campo total $\Delta T (\mathbf{r})$ (Eq. \ref{eq:delta-T-tilde-approx}). Todos os valores estão em nT.}
	\label{fig:prolate}
\end{figure}

\begin{table}[h!]
	\begin{center}
		\begin{tabular}{lc}
			
			&  \\
			& \\
			& \\
			& \\
			& \\
			& \\ 
			& \\
			& \\
						& \\
						& \\
		\end{tabular}
	\end{center}
\end{table}

\begin{figure}[hbt!]
	\centering \includegraphics[width=16cm,height=16cm]{figures/ellipsoid_oblate}
	\caption[Campo gerado pelo elipsoide oblato definido na Tabela 4.3. Componentes A) $x$, B) $y$ e C) $z$ da indução magnética $\Delta \mathbf{B}(\mathbf{r})$ (Eq. \ref{eq:delta-T-tilde}). D) Anomalia de campo total $\Delta T (\mathbf{r})$ (Eq. \ref{eq:delta-T-tilde-approx}). Todos os valores estão em nT.]{Campo gerado pelo elipsoide oblato definido na Tabela 4.3. Componentes A) $x$, B) $y$ e C) $z$ da indução magnética $\Delta \mathbf{B}(\mathbf{r})$ (Eq. \ref{eq:delta-T-tilde}). D) Anomalia de campo total $\Delta T (\mathbf{r})$ (Eq. \ref{eq:delta-T-tilde-approx}). Todos os valores estão em nT.}
	\label{fig:oblate}
\end{figure}

As Figuras \ref{fig:ellipsoid_triaxial_multi}A), \ref{fig:ellipsoid_triaxial_multi}B), \ref{fig:ellipsoid_triaxial_multi}C) e \ref{fig:ellipsoid_triaxial_multi}D) mostram, respectivamente, as componentes $x$, $y$ e $z$ da indução magnética $\Delta \mathbf{B}(\mathbf{r})$ (Eq. \ref{eq:delta-T-tilde}) e a
anomalia de campo total $\Delta T (\mathbf{r})$ (Eq. \ref{eq:delta-T-tilde-approx}) produzidas por dois elipsoides triaxiais definidos de acordo com os parâmetros da Tabela 4.4.

\newpage

\begin{table}[h!]
	\begin{center}
		\begin{tabular}{|l|c|c|}
			\hline
			\textbf{Parâmetro}  & \textbf{Valor}  & \textbf{Unidade}\\
			\hline 
			$a$, $b$, $c$   & 150, 100, 75 & m, m, m\\
			\hline
			\textit{strike}   & $0$ & $^{\circ}$\\
			\hline
			\textit{dip}    & $0$ & $^{\circ}$\\
			\hline
			\textit{rake}   & $0$  & $^{\circ}$\\
			\hline
			$x_c$   & -2500  & m\\
			\hline          
			$y_c$   & -2500  & m\\
			\hline                
			$z_c$  & 1000  & m\\
			\hline
			$\mathbf{M}_{R}$*  & 100, $25$, $40$  & $A/m$, $^{\circ}$, $^{\circ}$\\
			\hline
			$k_1$, $k_2$, $k_3$   & 0,01, 0,01, 0,01  & SI, SI, SI\\
			\hline
			Orientação das colunas da matriz $\mathbf{U}$**   & $0$, $90$, $90$  & $^{\circ}$, $^{\circ}$, $^{\circ}$\\
			\hline 
			$a_2$, $b_2$, $c_2$   & 200, 120, 60 & m, m, m\\
			\hline
			\textit{strike}$_2$   & $0$ & $^{\circ}$\\
			\hline
			\textit{dip}$_2$    & $0$ & $^{\circ}$\\
			\hline
			\textit{rake}$_2$   & $0$  & $^{\circ}$\\
			\hline
			$x_{c2}$   & 2500  & m\\
			\hline          
			$y_{c2}$   & 2500  & m\\
			\hline                
			$z_{c2}$   & 750  & m\\
			\hline
			$\mathbf{M}_{R2}$*  & 75, $25$, $40$  & $A/m$, $^{\circ}$, $^{\circ}$\\
			\hline
			$k_12$, $k_22$, $k_32$   & 0,01, 0,01, 0,01  & SI, SI, SI\\
			\hline
			Orientação das colunas da matriz $\mathbf{U}_2$**   & $0$, $90$, $90$  & $^{\circ}$, $^{\circ}$, $^{\circ}$\\
			\hline
			$\mathbf{B}_{0}$*    & 60000, $50$, $20$ & $nT$, $^{\circ}$, $^{\circ}$\\
			\hline			
		\end{tabular}
		\caption{Parâmetros dos  dois elipsoides triaxiais que produzem o campo magnético mostrado na Figura \ref{fig:ellipsoid_triaxial_multi}. *Valores de intensidade, inclinação e declinação respectivamente. **Ângulos de \textit{strike}, \textit{dip}  e \textit{rake}, respectivamente, utilizados para calcular os vetores unitários $\mathbf{u}_{1}$, $\mathbf{u}_{2}$, $\mathbf{u}_{3}$ por meio das Eqs. \ref{eq:v1_triaxial_prolate}, \ref{eq:v2_triaxial_prolate} e \ref{eq:v3_triaxial_prolate}.}
	\end{center}
	\label{tab:ellipsoid_triaxial_multi}
\end{table}

\begin{table}[h!]
	\begin{center}
		\begin{tabular}{lc}
			
			&  \\
			& \\
			&  \\
			& \\
			&  \\
			& \\
			&  \\
			& \\
			&  \\
			& \\
			&  \\
			& \\
			&  \\
			& \\
			&  \\
			& \\
		\end{tabular}
	\end{center}
\end{table}

\begin{figure}[hbt!]
	\centering \includegraphics[width=16cm,height=16cm]{figures/ellipsoid_triaxial_multi}
	\caption[Campo gerado pelos elipsoides triaxiais definido na Tabela 4.4. Componentes A) $x$, B) $y$ e C) $z$ da indução magnética $\Delta \mathbf{B}(\mathbf{r})$ (Eq. \ref{eq:delta-T-tilde}). D) Anomalia de campo total $\Delta T (\mathbf{r})$ (Eq. \ref{eq:delta-T-tilde-approx}). Todos os valores estão em nT.]{Campo gerado pelos elipsoides triaxiais definido na Tabela 4.4. Componentes A) $x$, B) $y$ e C) $z$, da indução magnética $\Delta \mathbf{B}(\mathbf{r})$ (Eq. \ref{eq:delta-T-tilde}). D) Anomalia de campo total $\Delta T (\mathbf{r})$ (Eq. \ref{eq:delta-T-tilde-approx}). Todos os valores estão em nT.}
	\label{fig:ellipsoid_triaxial_multi}
\end{figure}


A Figura \ref{fig:mag_fields_0} contém os \textit{scripts} com os parâmetros usados para a modelagem dos exemplos mostrados acima.

\begin{figure}[hbt!]
	\centering \includegraphics[width=16 cm,height=20 cm]{figures/mag_fields_011}
	\caption[\textit{Scripts} utilizados para gerar os resultados mostrados: A) na Figura \ref{fig:triaxial}, B) na Figura \ref{fig:prolate}, C) na Figura \ref{fig:oblate} e D) na Figura \ref{fig:ellipsoid_triaxial_multi}. A descrição do código é similar àquela mostrada na Figura \ref{fig:Cookbook_Triaxial}.]{\textit{Scripts} utilizados para gerar os resultados mostrados: A) na Figura \ref{fig:triaxial}, B) na Figura\ref{fig:prolate}, C) na Figura \ref{fig:oblate} e D) na Figura \ref{fig:ellipsoid_triaxial_multi}. A descrição do código é similar àquela mostrada na Figura \ref{fig:Cookbook_Triaxial}.}
	\label{fig:mag_fields_0}
\end{figure}

\begin{table}[h!]
	\begin{center}
		\begin{tabular}{lc}
			
			& \\
			& \\
			& \\
			& \\
			& \\
			& \\

		\end{tabular}
	\end{center}
\end{table}

\section{Fatores de desmagnetização}

Os fatores de desmagnetização são importantíssimos para a modelagem de corpos com alta susceptibilidade e dependem apenas da sua forma geométrica. 
Nesta seção, apresento simulações numéricas que visam ilustrar o comportamento dos fatores de desmagnetização em função do tamanho dos semi-eixos de elipsoides triaxiais, prolatos e oblatos. A partir de um elipsoide triaxial com semi-eixos $a_0=500$ m, $b_0=100$ m e $c_0=50$ m, gerei 200 outros elipsoides triaxiais com os semi-eixos destes dados por $a=a_0+u$, $b=b_0+u$ e $c=c_0+u$, em que $500 \le u \le 30000$. Note que, quando a variável $u$ assume valores pequenos, os elipsoides têm formato mais próximo ao elipsoide original. Por outro lado, quanto maior essa variável, mais esféricos serão os elipsoides obtidos. A Figura \ref{fig:n_triaxial} mostra os fatores de desmagnetização $\tilde{n}^{\dagger}_{11}$ (em azul), $\tilde{n}^{\dagger}_{22}$ (em verde) e $\tilde{n}^{\dagger}_{33}$ (em vermelho) calculados pelas Eqs. \ref{eq:n-tilde-dagger-11-triaxial}, \ref{eq:n-tilde-dagger-22-triaxial} e \ref{eq:n-tilde-dagger-33-triaxial}, respectivamente, para este conjunto de 200 elipsoides em função da variável $u$. Os resultados mostram que, para altos valores de $u$, todos os fatores de desmagnetização tendem ao valor de 1/3, que é típico de uma esfera. Por outro lado, quando a variável $u$ assume valores pequenos, os fatores de desmagnetização são muito diferentes uns dos outros.

\begin{figure}[hbt!]
	\centering \includegraphics[width=15cm,height=10cm]{figures/test_n_triaxial}
	\caption[Comparação entre os fatores de desmagnetização
	$\tilde{n}^{\dagger}_{11}$ (azul), $\tilde{n}^{\dagger}_{22}$ (verde) e $\tilde{n}^{\dagger}_{33}$ (vermelho) de um conjunto de 200 elipsoides triaxiais com semi-eixos $a = a_0 + u$, $b = b_0 + u$ e $c = c_0 + u$, em que $500 \le u \le 30000$, $a_0 = 500$ m, $b_0 = 100$ m e $c_0 = 50$ m. A linha preta horizontal tracejada representa o valor $1/3$. Os fatores de desmagnetização foram calculados, respectivamente, pelas Eqs. \ref{eq:n-tilde-dagger-11-triaxial}, \ref{eq:n-tilde-dagger-22-triaxial} e \ref{eq:n-tilde-dagger-33-triaxial}.]
	{Comparação entre os fatores de desmagnetização
		$\tilde{n}^{\dagger}_{11}$ (azul), $\tilde{n}^{\dagger}_{22}$ (verde) e $\tilde{n}^{\dagger}_{33}$ (vermelho) de um conjunto de 200 elipsoides triaxiais com semi-eixos $a = a_0 + u$, $b = b_0 + u$ e $c = c_0 + u$, em que $500 \le u \le 30000$, $a_0 = 500$ m, $b_0 = 100$ m e $c_0 = 50$ m. A linha preta horizontal tracejada representa o valor $1/3$. Os fatores de desmagnetização foram calculados, respectivamente, pelas Eqs. \ref{eq:n-tilde-dagger-11-triaxial}, \ref{eq:n-tilde-dagger-22-triaxial} e \ref{eq:n-tilde-dagger-33-triaxial}.}
	\label{fig:n_triaxial}
\end{figure}

Na Figura \ref{fig:n_prolato}, observamos o mesmo padrão dos fatores de desmagnetização para o caso de elipsoides prolatos. Neste caso, simulei 200 elipsoides prolatos com semi-eixos $a=m\, b_0$ e $b=b_0$, em que $b_0=100$ m e $1,1 \le m \le 100$ m. A Figura \ref{fig:n_prolato} mostra um gráfico dos seus fatores de desmagnetização $\tilde{n}^{\dagger}_{11}$ (Eq. \ref{eq:n-tilde-dagger-11-prolate}) e $\tilde{n}^{\dagger}_{22}$ (Eq. \ref{eq:n-tilde-dagger-22-prolate}) em função da variável $m$.
Para valores de $m$ próximos a 1, em que os semi-eixos têm tamanhos próximos, os fatores de desmagnetização tendem à $1/3$. Por outro lado, à medida em que $m$ aumenta, o semi-eixo $a$ fica muito maior que $b$ e os fatores de desmagnetização ficam muito diferentes um do outro.

\begin{figure}[hbt!]
	\centering \includegraphics[width=15cm,height=10cm]{figures/test_n_prolate}
	\caption[Comparação entre os fatores de desmagnetização $\tilde{n}^{\dagger}_{11}$ (azul) e $\tilde{n}^{\dagger}_{22}$ (verde) de um conjunto de 200 elipsoides prolatos com semi-eixos $a=m\, b_0$ e $b=b_0$, em que $b_0=100$ m e $1,1 \le m \le 100$ m. A linha preta horizontal tracejada representa o valor $1/3$. Os fatores de desmagnetização foram calculados, respectivamente pelas Eq. \ref{eq:n-tilde-dagger-11-prolate}, \ref{eq:n-tilde-dagger-22-prolate}.]{Comparação entre os fatores de desmagnetização $\tilde{n}^{\dagger}_{11}$ (azul) e $\tilde{n}^{\dagger}_{22}$ (verde) de um conjunto de 200 elipsoides prolatos com semi-eixos $a=m\, b_0$ e $b=b_0$, em que $b_0=100$ m e $1,1 \le m \le 100$ m. A linha preta horizontal tracejada representa o valor $1/3$. Os fatores de desmagnetização foram calculados, respectivamente pelas Eq. \ref{eq:n-tilde-dagger-11-prolate}, \ref{eq:n-tilde-dagger-22-prolate}.}
	\label{fig:n_prolato}
\end{figure}

De forma similar, simulei 200 elipsoides oblatos com semi-eixos $a=m\, b_0$ e $b=b_0$, em que $b_0=1000$ m e $0,05 \le m \le 1$ e calculei seus fatores de desmagnetização $\tilde{n}^{\dagger}_{11}$ e $\tilde{n}^{\dagger}_{22}$ utilizando, respectivamente, as Eqs. \ref{eq:n-tilde-dagger-11-oblate}, \ref{eq:n-tilde-dagger-22-oblate}. Lembre que, caso de elipsoides oblatos, $a$ é o semi-eixo menor e $b$ o semi-eixo maior. A Figura \ref{fig:n_oblato} mostra o resultado dessa simulação, para este conjunto de 200 elipsoides em função da variável $m$. Os resultados mostram que, para pequenos valores de $m$, que representa elipsoides com semi-eixos muito diferentes entre si, os fatores de desmagnetização são muito diferentes um do outro. Por outro lado, quando a variável $m$ assume valores mais altos, todos os fatores de desmagnetização tendem ao valor de 1/3, que é típico de uma esfera.\\

\begin{figure}[hbt!]
	\centering \includegraphics[width=15cm,height=10cm]{figures/test_n_oblate}
	\caption[Comparação entre os fatores de desmagnetização $\tilde{n}^{\dagger}_{11}$ (azul) e $\tilde{n}^{\dagger}_{22}$ (verde) de um conjunto de 200 elipsoides oblatos com semi-eixos $a=m\, b_0$ e $b=b_0$, em que $b_0=1000$ m e $0,05 \le m \le 1$. A linha preta horizontal tracejada representa o valor $1/3$. Os fatores de desmagnetização foram calculados, respectivamente pelas Eq. \ref{eq:n-tilde-dagger-11-oblate}, \ref{eq:n-tilde-dagger-22-oblate}.]{Comparação entre os fatores de desmagnetização $\tilde{n}^{\dagger}_{11}$ (azul) e $\tilde{n}^{\dagger}_{22}$ (verde) de um conjunto de 200 elipsoides oblatos com semi-eixos $a=m\, b_0$ e $b=b_0$, em que $b_0=1000$ m e $0,05 \le m \le 1$. A linha preta horizontal tracejada representa o valor $1/3$. Os fatores de desmagnetização foram calculados, respectivamente pelas Eq. \ref{eq:n-tilde-dagger-11-oblate}, \ref{eq:n-tilde-dagger-22-oblate}.}
	\label{fig:n_oblato}
\end{figure}

Através destes gráficos é possível observar que os fatores de desmagnetização possuem valor menor, quanto maior o semi-eixo do elipsoide. Fisicamente, isto significa que há a tendência do elipsoide se desmagnetizar com maior intensidade na direção dos seus semi-eixos menores.
\newpage

\section{Comparações entre a $\Delta T (\mathbf{r})$ produzida por modelos elipsoidais similares}

Nesta seção é apresentada uma série de comparações entre os elipsoides e implementações diversas afim de validação dos resultados.
A Figura \ref{fig:triaxial_sphere} mostra as anomalias de campo total $\Delta T (\mathbf{r})$ \ref{eq:delta-T-tilde-approx} produzidas pelo elipsoide triaxial definido na Tabela 4.5 e por uma esfera. Os dados produzidos pela esfera foram calculados com o pacote \textit{Fatiando a Terra}.

\begin{table}[h!]
	\begin{center}
		\begin{tabular}{lc}
			
			&  \\
			& \\

		\end{tabular}
	\end{center}
\end{table}

\begin{table}[h!]
	\begin{center}
		\begin{tabular}{|l|c|c|}
			\hline
			\textbf{Parâmetro}  & \textbf{Valor}  & \textbf{Unidade}\\
			\hline 
			$a$, $b$, $c$   & 500,0001, 500,0, 499,9999   & m\\
			\hline
			\textit{strike}   & $0$ & $^{\circ}$\\
			\hline
			\textit{dip}    & $0$ & $^{\circ}$\\
			\hline
			\textit{rake}   & $0$  & $^{\circ}$\\
			\hline
			$x_c $   & 0  & m\\
			\hline          
			$y_c $   & 0  & m\\
			\hline                
			$z_c $   & 1000  & m\\
			\hline
			$\mathbf{M}_{R}$*  & 100, $25$, $40$  & $A/m$, $^{\circ}$, $^{\circ}$\\
			\hline
			$\mathbf{B}_{0}$*    & 1, $50$, $20$ & $nT$, $^{\circ}$, $^{\circ}$\\
			\hline
			$k_{1}$, $k_{2}$, $k_{3}$   & 0,1, 0,1, 0,1 & SI, SI, SI \\
			\hline
			Orientação das colunas da matriz $\mathbf{U}$**   & $0$, $90$, $90$  & $^{\circ}$, $^{\circ}$, $^{\circ}$\\
			\hline
		\end{tabular}
		\caption{Parâmetros do elipsoide triaxial que produz o campo magnético mostrado na Figura \ref{fig:triaxial_sphere}A. *Valores de intensidade, inclinação e declinação respectivamente. **Ângulos de \textit{strike}, \textit{dip}  e \textit{rake}, respectivamente, utilizados para calcular os vetores unitários $\mathbf{u}_{1}$, $\mathbf{u}_{2}$, $\mathbf{u}_{3}$ por meio das Eqs. \ref{eq:v1_triaxial_prolate}, \ref{eq:v2_triaxial_prolate} e \ref{eq:v3_triaxial_prolate}.}
	\end{center}
	\label{tab:triaxial_sphere}
\end{table}

\begin{table}[h!]
	\begin{center}
		\begin{tabular}{lc}
			
			&  \\
			& \\
			& \\
			&  \\
				& \\
				& \\
				&  \\
					& \\
				& \\
				&  \\
				& \\
				& \\
		\end{tabular}
	\end{center}
\end{table}

\begin{figure}[hbt!]
	\centering \includegraphics[width=13.0 cm,height=20 cm]{figures/ellipsoid_triaxial_sphere}
	\caption[Comparação entre a anomalia de campo total $\Delta T (\mathbf{r})$ (Eq. \ref{eq:delta-T-tilde-approx}) produzida (A) pelo elipsoide triaxial definido na Tabela 4.5 e (B) por uma esfera com centro em $(0, 0, 1000)$ e raio $500$ m. Os dois corpos sintéticos possuem a mesma magnetização $\mathbf{M}$ (Eq. \ref{eq:M-K-isotropic}). A anomalia produzida pela esfera foi calculada utilizando-se o pacote \textit{Fatiando a Terra}. Os dados estão calculados em uma malha regular de $200 \times 200$ pontos, no plano horizontal $z = 0$ m.]{Comparação entre a anomalia de campo total $\Delta T (\mathbf{r})$ (Eq. \ref{eq:delta-T-tilde-approx}) produzida (A) pelo elipsoide triaxial definido na Tabela 4.5 e (B) por uma esfera com centro em $(0, 0, 1000)$ e raio $500$ m. Os dois corpos sintéticos possuem a mesma magnetização $\mathbf{M}$ (Eq. \ref{eq:M-K-isotropic}). A anomalia produzida pela esfera foi calculada utilizando-se o pacote \textit{Fatiando a Terra}. Os dados estão calculados em uma malha regular de $200 \times 200$ pontos, no plano horizontal $z = 0$ m.}
	\label{fig:triaxial_sphere}
\end{figure}

Tal como esperado, as anomalias de campo total produzidas pelo elipsoide triaxial (Fig. \ref{fig:triaxial_sphere}A) e pela esfera (Fig. \ref{fig:triaxial_sphere}B) são muito próximas. As diferenças máximas ocorrem nas regiões próximas de máximo e mínimo das anomalias e atingem um valor absoluto de $\sim 0.000349$ nT. Estes resultados representam uma validação numérica das rotinas desenvolvidas neste trabalho para calcular a anomalia de campo total produzida por elipsoides triaxiais.

As Figuras \ref{fig:triaxial_prolate}A e \ref{fig:triaxial_prolate}B mostram as anomalias de campo total $\Delta T (\mathbf{r})$ (Eq. \ref{eq:delta-T-tilde-approx}) produzidas, respectivamente, pelo elipsoide triaxial definido na Tabela 4.6 e pelo elipsoide prolato definido na Tabela 4.7. As maiores diferenças ocorrem próximas as região de máximo e minimo das anomalias e atingem um valor absoluto de $\sim 0.003788$ nT. Similarmente, as Figuras \ref{fig:triaxial_oblate}A e \ref{fig:triaxial_oblate}B mostram as anomalias de campo total $\Delta T (\mathbf{r})$ (Eq. \ref{eq:delta-T-tilde-approx}) produzidas, respectivamente, pelo elipsoide triaxial definido na Tabela 4.8 e pelo elipsoide oblato definido na Tabela 4.9. As maiores diferenças ocorrem na região de máximo/minimo das anomalias e atingem um valor absoluto de $\sim 46.71$ nT, o que representa cerca de $0,5\%$ do valor máximo da anomalia apresentada nesta simulação. Estes resultados servem como validação numérica e mostram a consistência entre as rotinas desenvolvidas neste trabalho.
\newpage

\begin{table}[h!]
	\begin{center}
		\begin{tabular}{|l|c|c|}
			\hline
			\textbf{Parâmetro}  & \textbf{Valor}  & \textbf{Unidade} \\
			\hline 
			$a$, $b$, $c$  & 500, 100, 99.99 & m\\
			\hline
			\textit{strike}   & $90$ & $^{\circ}$\\
			\hline
			\textit{dip}    & $45$ & $^{\circ}$\\
			\hline
			\textit{rake}   & $0$  & $^{\circ}$\\
			\hline
			$x_c$   & 0 & m \\
			\hline          
			$y_c$   & 0  & m\\
			\hline                
			$z_c$   & 1000  & m\\
			\hline
			$\mathbf{M}_{R}$*  & 100, $90$, $0$  & $A/m$, $^{\circ}$, $^{\circ}$\\
			\hline
			$\mathbf{B}_{0}$*    & 60000, $50$, $20$ & $nT$, $^{\circ}$, $^{\circ}$\\
			\hline
			$k_{1}$, $k_{2}$, $k_{3}$   & 0,2, 0,1, 0,05  & SI, SI, SI\\
			\hline
			Orientação das colunas da matriz $\mathbf{U}$**   & $0$, $90$, $90$  & $^{\circ}$, $^{\circ}$, $^{\circ}$\\
			\hline
		\end{tabular}
		\caption{Parâmetros do elipsoide triaxial que produz o campo magnético mostrado na Figura \ref{fig:triaxial_prolate}A. *Valores de intensidade, inclinação e declinação respectivamente. **Ângulos de \textit{strike}, \textit{dip}  e \textit{rake}, respectivamente, utilizados para calcular os vetores unitários $\mathbf{u}_{1}$, $\mathbf{u}_{2}$, $\mathbf{u}_{3}$ por meio das Eqs. \ref{eq:v1_triaxial_prolate}, \ref{eq:v2_triaxial_prolate} e \ref{eq:v3_triaxial_prolate}.}
	\end{center}
	\label{tab:triaxial_prolate1}
\end{table}

\vspace{2cm}

\begin{table}[h!]
	\begin{center}
		\begin{tabular}{|l|c|c|}
			\hline
			\textbf{Parâmetro}  & \textbf{Valor}  & \textbf{Unidade}\\
			\hline 
			$a$, $b$  & 500, 100 & m\\
			\hline
			\textit{strike}   & $90$ & $^{\circ}$\\
			\hline
			\textit{dip}    & $45$ & $^{\circ}$\\
			\hline
			\textit{rake}   & $0$  & $^{\circ}$\\
			\hline
			$x_c$   & 0  & m\\
			\hline          
			$y_c$   & 0  & m\\
			\hline                
			$z_c$  & 1000  & m\\
			\hline
			$\mathbf{M}_{R}$*  & 100, $90$, $0$  & $A/m$, $^{\circ}$, $^{\circ}$\\
			\hline
			$\mathbf{B}_{0}$*    & 60000, $50$, $20$ & $nT$, $^{\circ}$, $^{\circ}$\\
			\hline
			$k_{1}$, $k_{2}$, $k_{3}$   & 0,2, 0,1, 0,05  & SI, SI, SI\\
			\hline
			Orientação das colunas da matriz $\mathbf{U}$**   & $0$, $90$, $90$  & $^{\circ}$, $^{\circ}$, $^{\circ}$\\
			\hline
		\end{tabular}
		\caption{Parâmetros do elipsoide prolato que produz o campo magnético mostrado na Figura \ref{fig:triaxial_prolate}B. *Valores de intensidade, inclinação e declinação respectivamente. **Ângulos de \textit{strike}, \textit{dip}  e \textit{rake}, respectivamente, utilizados para calcular os vetores unitários $\mathbf{u}_{1}$, $\mathbf{u}_{2}$, $\mathbf{u}_{3}$ por meio das Eqs. \ref{eq:v1_triaxial_prolate}, \ref{eq:v2_triaxial_prolate} e \ref{eq:v3_triaxial_prolate}.}
	\end{center}
	\label{tab:triaxial_prolate2}
\end{table}

\begin{figure}[hbt!]
	\centering \includegraphics[width=14.5 cm,height=22 cm]{figures/ellipsoid_triaxial_prolate}
	\caption[Comparação entre a anomalia de campo total $\Delta T (\mathbf{r})$ (Eq. \ref{eq:delta-T-tilde-approx}) produzida (A) pelo elipsoide triaxial definido na Tabela 4.6 e (B) produzida pelo elipsoide prolato definido na Tabela 4.7.]{Comparação entre a anomalia de campo total $\Delta T (\mathbf{r})$ (Eq. \ref{eq:delta-T-tilde-approx}) produzida (A) pelo elipsoide triaxial definido na Tabela 4.6 e (B) produzida pelo elipsoide prolato definido na Tabela 4.7.}
	\label{fig:triaxial_prolate}
\end{figure}

\begin{table}[h!]
	\begin{center}
		\begin{tabular}{lc}
			
			&  \\
			& \\
			& \\
		\end{tabular}
	\end{center}
\end{table}

\begin{table}[h!]
	\begin{center}
		\begin{tabular}{|l|c|c|}
			\hline
			\textbf{Parâmetro}  & \textbf{Valor} & \textbf{Unidade} \\
			\hline 
			$a$, $b$, $c$   & 500, 499.99, 499.98 & m\\
			\hline
			\textit{strike}   & $0$ & $^{\circ}$\\
			\hline
			\textit{dip}   & $0$ & $^{\circ}$\\
			\hline
			\textit{rake}   & $90$  & $^{\circ}$\\
			\hline
			$x_c$   & 0  & m\\
			\hline          
			$y_c$   & 0  & m\\
			\hline                
			$z_c$   & 1000 & m \\
			\hline
			$\mathbf{M}_{R}$*  & 100, $90$, $0$ & $A/m$, $^{\circ}$, $^{\circ}$ \\
			\hline
			$\mathbf{B}_{0}$*    & 60000, $50$, $20$ & $nT$, $^{\circ}$, $^{\circ}$ \\
			\hline
			$k_{1}$, $k_{2}$, $k_{3}$   & 0,1, 0,1, 0,1 & SI, SI, SI \\
			\hline
			Orientação das colunas da matriz $\mathbf{U}$**  & $0$, $90$, $90$ & $^{\circ}$, $^{\circ}$, $^{\circ}$ \\
			\hline
		\end{tabular}
		\caption{Parâmetros do elipsoide triaxial que produz o campo magnético mostrado na Figura \ref{fig:triaxial_oblate}A. *Valores de intensidade, inclinação e declinação respectivamente. **Ângulos de \textit{strike}, \textit{dip} e \textit{rake}, respectivamente, utilizados para calcular os vetores unitários $\mathbf{u}_{1}$, $\mathbf{u}_{2}$, $\mathbf{u}_{3}$ por meio das Eqs. \ref{eq:v1_triaxial_prolate}, \ref{eq:v2_triaxial_prolate} e \ref{eq:v3_triaxial_prolate}.}
	\end{center}
	\label{tab:triaxial_oblate1}
\end{table}

\begin{table}[h!]
	\begin{center}
		\begin{tabular}{lc}
			
			&  \\
			& \\
			& \\
			&  \\
			& \\
			&  \\
			& \\
		\end{tabular}
	\end{center}
\end{table}

\begin{table}[h!]
	\begin{center}
		\begin{tabular}{|l|c|c|}
			\hline
			\textbf{Parâmetro}  & \textbf{Valor} & \textbf{Unidade} \\
			\hline 
			$a$, $b$   & 499,99, 500 & m\\
			\hline
			\textit{strike}   & $0$ & $^{\circ}$\\
			\hline
			\textit{dip}    & $0$ & $^{\circ}$\\
			\hline
			\textit{rake}   & $90$  & $^{\circ}$\\
			\hline
			$x_c$   & 0  & m\\
			\hline          
			$y_c$   & 0  & m\\
			\hline                
			$z_c$   & 1000  & m\\
			\hline
			$\mathbf{M}_{R}$*  & 100, $90$, $0$  & $A/m$, $^{\circ}$, $^{\circ}$\\
			\hline
			$\mathbf{B}_{0}$*    & 60000, $50$, $20$ & $nT$, $^{\circ}$, $^{\circ}$ \\
			\hline
			$k_{1}$, $k_{2}$, $k_{3}$   & 0,1, 0,1, 0,1 & SI, SI, SI \\
			\hline
			Orientação das colunas da matriz $\mathbf{U}$**   & $0$, $90$, $90$ & $^{\circ}$, $^{\circ}$, $^{\circ}$ \\
			\hline
		\end{tabular}
		\caption{Parâmetros do elipsoide oblato que produz o campo magnético mostrado na Figura \ref{fig:triaxial_oblate}B. *Valores de intensidade, inclinação e declinação respectivamente. **Ângulos de \textit{strike}, \textit{dip}  e \textit{rake}, respectivamente, utilizados para calcular os vetores unitários $\mathbf{u}_{1}$, $\mathbf{u}_{2}$, $\mathbf{u}_{3}$ por meio das Eqs. \ref{eq:v1_oblate}, \ref{eq:v2_oblate} e \ref{eq:v3_oblate}.}
	\end{center}
	\label{tab:triaxial_oblate2}
\end{table}

\begin{figure}[hbt!]
	\centering \includegraphics[width=14.5 cm,height=22 cm]{figures/ellipsoid_triaxial_oblate}
	\caption[Comparação entre a anomalia de campo total $\Delta T (\mathbf{r})$ (Eq. \ref{eq:delta-T-tilde-approx}) produzida (A) pelo elipsoide triaxial definido na Tabela 4.8 e (B) produzida pelo elipsoide oblato definido na Tabela 4.9.]{Comparação entre a anomalia de campo total $\Delta T (\mathbf{r})$ (Eq. \ref{eq:delta-T-tilde-approx}) produzida (A) pelo elipsoide triaxial definido na Tabela 4.8 e (B) produzida pelo elipsoide oblato definido na Tabela 4.9.}
	\label{fig:triaxial_oblate}
\end{figure}

\section{Susceptibilidade}

Sabe-se que, na modelagem magnética de corpos geológicos com alta susceptibilidade, a desmagnetização é um fator muito importante. Nestes casos, negligenciar a desmagnetização e, consequentemente, utilizar a magnetização aproximada $\breve{\mathbf{M}}$ (Eq. \ref{eq:M-approx}) ao invés da magnetização $\mathbf{M}$ (Eq. \ref{eq:M-K-isotropic}), pode acarretar em erros de interpretação das anomalias em dados magnéticos. É comumente aceito na literatura que 0,1 SI é o valor de suscetibilidade isotrópica a partir do qual a desmagnetização deve ser considerada na modelagem ou seja, é o valor a partir do qual a magnetização aproximada $\breve{\mathbf{M}}$ (Eq. \ref{eq:M-approx}) não é considerada uma boa aproximação para a magnetização $\mathbf{M}$ (Eq. \ref{eq:M-K-isotropic}). Para suscetibilidades menores que 0,1 SI, considera-se que $\breve{\mathbf{M}}$ (Eq. \ref{eq:M-approx}) é uma boa aproximação para a magnetização resultante e que a parcela induzida tem a mesma direção do campo geomagnético local.

Para analisar o efeito da suscetibilidade na modelagem magnética, calculamos a magnetização resultante $\mathbf{M}$ (Eq. \ref{eq:M-K-isotropic}) do elipsoide triaxial definido na Tabela 4.10 para diferentes valores de suscetibilidade isotrópica $0,0001 \le \chi \le 0,05$. A Figura \ref{fig:test_k_triaxial} mostra a inclinação (linha azul) e a declinação (linha verde) da magnetização resultante $\mathbf{M}$ (Eq. \ref{eq:M-K-isotropic}) em função da suscetibilidade $\chi$. A inclinação e a declinação do campo geomagnético local estão representadas, respectivamente, pelas linhas horizontais preta e vermelha. Nota-se que, para valores de suscetibilidade $\chi$ menores que 0,1 SI (linha preta vertical na Figura \ref{fig:test_k_triaxial}), a diferença entre a inclinação da magnetização resultante do elipsoide e do campo geomagnético local é menor que $-0,58$ graus. Já para a declinação, a diferença é menor que $0,85$ graus. Por outro lado, as diferenças na inclinação e declinação para valores de suscetibilidade $\chi = 0,5$ SI são próximos a $-2,71$ e $3,4$ graus, respectivamente. A linha vertical amarela na Figura \ref{fig:test_k_triaxial} mostra a suscetibilidade $\chi_{max}$ (Eq. \ref{eq:max-chi}) predita para que o erro relativo na magnetização resultante deste elipsoide seja menor ou igual a $\epsilon = 5\%$. Neste caso, a suscetibilidade $\chi_{max}$ ($0,077$ SI) é menor que aquela comumente utilizada na literatura (0,1 SI), as diferenças na inclinação e declinação são iguais a $-0,46$ e $0,68$ graus, respectivamente, e o erro relativo da magnetização é igual a $1,593\%$. Este resultado mostra que o valor de suscetibilidade isotrópica a partir do qual a desmagnetização deve ser considerada na modelagem pode ser escolhido de acordo com dois critérios: i) o conhecimento prévio do intérprete sobre a forma do corpo e ii) o máximo erro relativo $\epsilon$ permitido para a magnetização resultante.

\newpage

\begin{table}[h!]
	\begin{center}
		\begin{tabular}{|l|c|c|}
			\hline
			\textbf{Parâmetro}  & \textbf{Valor} & \textbf{Unidade} \\
			\hline 
			$a$, $b$, $c$   & 500, 100, 50 & m, m, m\\
			\hline
			\textit{strike}   & $0$ & $^{\circ}$\\
			\hline
			\textit{dip}   & $0$ & $^{\circ}$\\
			\hline
			\textit{rake}   & $0$  & $^{\circ}$\\
			\hline
			$x_c$   & 0  & m\\
			\hline          
			$y_c$   & 0  & m\\
			\hline                
			$z_c$   & 500 & m \\
			\hline
			$\mathbf{M}_{R}$*  & 0, $0$, $0$ & $A/m$, $^{\circ}$, $^{\circ}$ \\
			\hline
			$\mathbf{B}_{0}$*    & 60000, $30$, $-15$ & $nT$, $^{\circ}$, $^{\circ}$ \\
			\hline
			Orientação das colunas da matriz $\mathbf{U}$**  & $0$, $90$, $90$ & $^{\circ}$, $^{\circ}$, $^{\circ}$ \\
			\hline
		\end{tabular}
		\caption{Parâmetros do elipsoide triaxial que produz o campo magnético mostrado na Figura \ref{fig:test_k_triaxial}. *Valores de intensidade, inclinação e declinação respectivamente. **Ângulos de \textit{strike}, \textit{dip}  e \textit{rake}, respectivamente, utilizados para calcular os vetores unitários $\mathbf{u}_{1}$, $\mathbf{u}_{2}$, $\mathbf{u}_{3}$ por meio das Eqs. \ref{eq:v1_triaxial_prolate}, \ref{eq:v2_triaxial_prolate} e \ref{eq:v3_triaxial_prolate}.}
	\end{center}
	\label{tab:triaxial_k_iso}
\end{table}

\begin{figure}[hbt!]
	\centering \includegraphics[width=13.5 cm,height=9 cm]{figures/test_k_triaxial}
	\caption[Efeito da susceptibilidade isotrópica $\chi$ na modelagem do elipsoide triaxial definido na Tabela 4.10. As linhas azul e verde representam, respectivamente, a inclinação $I$ e a declinação $D$ da magnetização resultante $\mathbf{M}$ (Eq. \ref{eq:M-K-isotropic}) deste elipsoide em função da suscetibilidade isotrópica $\chi$. As linhas pontilhadas horizontais preta e vermelha representam, respectivamente, a inclinação $I_{0}$ e a declinação $D_{0}$ do campo geomagnético local (Tabela 4.10). A linha vertical preta representa o valor de suscetibilidade isotrópica $\chi = 0,1$ SI, que é comumente aceito como o limite a partir do qual a desmagnetização deve ser considerada na modelagem. A linha vertical amarela representa a suscetibilidade isotrópica $\chi_{max}$ calculado com a Eq. \ref{eq:max-chi} para que o erro relativo na magnetização seja menor ou igual a $\epsilon = 5 \%$.]
	{Efeito da susceptibilidade isotrópica $\chi$ na modelagem do elipsoide triaxial definido na Tabela 4.10. As linhas azul e verde representam, respectivamente, a inclinação $I$ e a declinação $D$ da magnetização resultante $\mathbf{M}$ (Eq. \ref{eq:M-K-isotropic}) deste elipsoide em função da suscetibilidade isotrópica $\chi$. As linhas pontilhadas horizontais preta e vermelha representam, respectivamente, a inclinação $I_{0}$ e a declinação $D_{0}$ do campo geomagnético local (Tabela 4.10). A linha vertical preta representa o valor de suscetibilidade isotrópica $\chi = 0,1$ SI, que é comumente aceito como o limite a partir do qual a desmagnetização deve ser considerada na modelagem. A linha vertical amarela representa a suscetibilidade isotrópica $\chi_{max}$ calculado com a Eq. \ref{eq:max-chi} para que o erro relativo na magnetização seja menor ou igual a $\epsilon = 5 \%$.}
	\label{fig:test_k_triaxial}
\end{figure}

\section{Anisotropia de forma}

Conforme mostrado na seção 4.2 existe uma relação entre os fatores de desmagnetização e os semi-eixos. Na Figura \ref{fig:ellipsoid_shape_iso10} mostramos como o aumento do semi-eixo maior afeta o vetor de magnetização resultante (Eq. \ref{eq:M-KH}). A princípio, os três semi-eixos estão muito próximos, simulando uma esfera. Quando postos sob um campo externo, o vetor de magnetização resultante se direciona para a direção deste campo. Porém a medida que aumenta-se o semi-eixo maior, ocorre a desmagnetização nas direções dos semi-eixos menores e o vetor de magnetização resultante tende a se alinhar na direção do semi-eixo maior (neste caso o elipsoide triaxial possui um azimute de 10$^{\circ}$).

\vspace{2cm}

\begin{table}[h!]
	\begin{center}
		\begin{tabular}{|l|c|c|}
			\hline
			\textbf{Parâmetro}  & \textbf{Valor}  & \textbf{Unidade }\\
			\hline 
			$a$, $b$, $c$  & 50,1-1000, 50, 49,9 & m, m, m\\
			\hline
			\textit{strike}   & $10$ & $^{\circ}$\\
			\hline
			\textit{dip}    & $0$ & $^{\circ}$\\
			\hline
			\textit{rake}   & $0$  & $^{\circ}$\\
			\hline
			$x_c$   & 0  & m\\
			\hline          
			$y_c$   & 0  & m\\
			\hline                
			$z_c$   & 1000  & m\\
			\hline
			$\mathbf{M}_{R}$*  & 100, $0$, $0$ & $A/m$, $^{\circ}$, $^{\circ}$ \\
			\hline
			$\mathbf{B}_{0}$*   & 60000, $0$, $90$ & $nT$, $^{\circ}$, $^{\circ}$\\
			\hline
			$k_{1}$, $k_{2}$, $k_{3}$   & 2, 2, 2  & SI, SI, SI\\
			\hline
			Orientação das colunas da matriz $\mathbf{U}$**   & $0$, $90$, $90$  & $^{\circ}$,  $^{\circ}$,  $^{\circ}$\\
			\hline
		\end{tabular}
		\caption{Parâmetros do elipsoide triaxial modelado. *Valores de intensidade, inclinação e declinação respectivamente. **Ângulos de \textit{strike}, \textit{dip}  e \textit{rake}, respectivamente, utilizados para calcular os vetores unitários $\mathbf{u}_{1}$, $\mathbf{u}_{2}$, $\mathbf{u}_{3}$ por meio das Eqs. \ref{eq:v1_triaxial_prolate}, \ref{eq:v2_triaxial_prolate} e \ref{eq:v3_triaxial_prolate}.}
	\end{center}
	\label{tab:ellipsoid_shape_iso10}
\end{table}

\begin{figure}[hbt!]
	\centering \includegraphics[width=16 cm,height=16 cm]{figures/ellipsoid_shape_iso}
	\caption[Simulação, da mudança do vetor de magnetização resultante (Eq. \ref{eq:M-KH}), de um elipsoide triaxial com o aumento do semi-eixo maior. O elipsoide está imerso em um campo externo constante de declinação $90^{\circ}$, possui susceptibilidade constante e direcionado com um azimute de $10^{\circ}$. Ao longo da sequência das figuras, seu semi-eixo maior aumenta de proporção em relação aos demais (variando entre 50 e 3000 m.). Nota-se a tendência do vetor de magnetização resultante (seta em vermelho) em se alinhar com o semi-eixo maior.]
	{Simulação, da mudança do vetor de magnetização resultante (Eq. \ref{eq:M-KH}), de um elipsoide triaxial com o aumento do semi-eixo maior. O elipsoide está imerso em um campo externo constante de declinação $90^{\circ}$, possui susceptibilidade constante e direcionado com um azimute de $10^{\circ}$. Ao longo da sequência das figuras, seu semi-eixo maior aumenta de proporção em relação aos demais (variando entre 50 e 3000 m.). Nota-se a tendência do vetor de magnetização resultante (seta em vermelho) em se alinhar com o semi-eixo maior.}
	\label{fig:ellipsoid_shape_iso10}
\end{figure}
