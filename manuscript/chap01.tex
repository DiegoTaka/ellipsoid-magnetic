\chapter{Introdução}


Baseado na teoria matemática da indução magnética desenvolvida por
\citet{poisson1824}, \citet{maxwell1873}, afirmou que se $U$ 
é o potencial gravitacional produzido por um corpo de 
densidade $\rho$ e forma geométrica arbitrária em um ponto $(x, y, z)$,
então $-\frac{\partial U}{\partial x}$ é o potencial magnético escalar que seria
produzido por este mesmo corpo, no mesmo ponto, se sua magnetização fosse 
uniforme ao longo do eixo $x$ e tivesse intensidade $\rho$.
\citet{maxwell1873}, generalizou essa ideia como uma forma de
determinar o potencial magnético escalar produzido por um corpo 
magnetizado uniformemente em qualquer direção. Presumindo que 
esta magnetização uniforme seria devida à indução e proporcional ao campo magnético resultante (intensidade) dentro do corpo, Maxwell postulou que este campo também deveria ser uniforme e paralelo à magnetização, uma vez que este é proporcional ao gradiente do potencial magnético escalar dentro do corpo.
Como consequência, o potencial gravitacional $U$ nos pontos dentro do
corpo deveria ser descrito por uma função quadrática das coordenadas espaciais.
Aparentemente, \citet{maxwell1873} foi o primeiro a postular que elipsoides
são os únicos corpos finitos que possuem um potencial gravitacional que
satisfaz essa propriedade e que, portanto, são os únicos que podem ser
magnetizados uniformemente na presença de um campo magnético uniforme.

A capacidade de ser magnetizado uniformemente na presença de um campo magnético uniforme 
pode ser estendida para outros corpos cuja forma se
deriva de um elipsoide (e.g., esferas, cilindros elípticos). Por outro lado, todos
os demais corpos não podem ser magnetizados uniformemente na presença de um
campo indutor uniforme \citep{jahren1963, schlomann1965, clark1999}.
Uma consequência importante proveniente da uniformidade do campo magnético interno em elipsoides, 
é que estes corpos são os únicos que possuem solução 
verdadeiramente analítica para a desmagnetização \citet{clark1986}.
A desmagnetização é o efeito produzido pelo campo que é criado no interior dos corpos (campo desmagnetizante) em resposta ao campo indutor. A desmagnetização contribui para o decréscimo da magnetização resultante e dependente apenas da forma do corpo, isto é, depende apenas dos seus semi-eixos. A desmagnetização dá origem a anisotropia magnética de forma. O termo anisotropia de forma pode ser definido como a existência de direções preferenciais de magnetização dentro do corpo, isto é, há direções em que o corpo é magnetizado com mais facilidade e direções em que é mais difícil \citep{thompson1986, dunlop1997, clark1999}.
Há outros dois tipos de anisotropia magnética: a anisotropia magnetocristalina e a anisotropia magnetostritiva. A anisotropia magnetocristalina é resultado da geometria interna e da composição química dos minerais que formam o corpo. Já a anisotropia de magnetostritiva, que ocorre quando as propriedades magnéticas do corpo são alteradas quando este é submetido \citep{tauxe2003rudiments, thompson1986}.

É comumente aceito pela comunidade científica, que a desmagnetização pode ser 
negligenciada se o corpo possui uma susceptibilidade menor que 0.1 SI
\citep{austin2014, clark2014, clark1986, emerson1985, eskola1980, guo1998, guo2001, purss2005, hillan2013}.
Também é comumente aceito que negligenciar a desmagnetização em corpos que possuem 
susceptibilidade maior que 0.1 SI, tal como formações ferríferas bandadas, pode comprometer drasticamente
os resultados obtidos por modelagem magnética.

Devido a flexibilidade  de parametrização, elipsoides podem assumir uma grande variedade de formas e assim serem 
usados para representar por exemplo, desde corpos esféricos (três semi-eixos similares) até corpos mineralizados em formato de $pipe$ (um semi-eixo mais alongado).
\citet{farrar1979} demonstrou a importância do modelo elipsoidal ao utiliza-lo para representar
em consideração a desmagnetização de forma adequada a determinar direções de 
perfuração confiáveis no campo de Tennant Creek, Austrália.
Posteriormente, \citet{hoschke1991} também mostrou a eficiência do
modelo elipsoidal para localizar e definir os limites de corpos mineralizados de ferro no campo de Tennant Creek.
\citet{clark2000} forneceu uma discussão detalhada sobre a influência da desmagnetização na interpretação magnética do depósito de cobre-ouro de Osborne, Austrália. Este depósito é hospedado por corpos mineralizados de ferro de alta susceptibilidade. De acordo com \citet{clark2000}, negligenciar a desmagnetização levaria à um erro de $\approx 55^{\circ}$ na interpretação do mergulho real da estrutura.
Baseado em modelagem magnética e em medidas de propriedades magnéticas, \citet{austin2014} mostraram que, ao contrário de interpretações anteriores, a magnetização do depósito de óxido de ferro-cobre-ouro (IOCG) em Candelaria, Chile, não é dominada pela componente da indução. Na verdade, o depósito tem uma fraca magnetização remanente e é fortemente afetada pela desmagnetização.

Estes exemplos mostram a importância prática da desmagnetização para a correta interpretação dos dados magnéticos produzidos por corpos geológicos de alta susceptibilidade.
Além disso, mostram a importância do modelo elipsoidal para produzir modelos geológicos confiáveis de corpos mineralizados, o que pode significar em uma redução significativa dos custos associados com perfuração.

Uma vasta literatura sobre a modelagem magnética de corpos elipsoidais foi desenvolvida ao longo dos anos por muitos pesquisadores. No entanto, é muito difícil encontrar, em um mesmo trabalho, os aspectos teóricos envolvidos na modelagem magnética de elipsoides triaxiais, prolatos e oblatos, que possuam susceptibilidade magnética isotrópica e anisotrópica e que esteja orientado de forma arbitrária. Além disso, falta para a comunidade de geociências uma ferramenta de fácil uso para simular o campo magnético produzido por elipsoides uniformemente magnetizados. Tal ferramenta seria útil tanto para ensino como para a pesquisa em geofísica de exploração. 

Nesta dissertação, apresento uma revisão teórica integrada sobre a modelagem magnética de corpos elipsoidais. A revisão considera corpos triaxiais, prolatos e oblatos, com susceptibilidade isotrópica e anisotrópica e com a presença de magnetização remanente. Além disso, apresento uma discussão sobre o valor de susceptibilidade isotrópica acima do qual a desmagnetização deve ser levada em consideração na modelagem. Este valor limite é definido com base na forma do corpo e no erro relativo máximo que o intérprete define para a magnetização resultante calculada na modelagem. Nesta dissertação, também apresento uma série de rotinas, escritas em linguagem \textit{Python} para modelar o campo magnético produzido por elipsoides. As rotinas foram baseadas no pacote \textit{Fatiando a Terra} \citep{uieda-proc-scipy-2013}, que é um projeto de código aberto e acesso livre para a modelagem e inversão em geofísica. Tentamos usar as melhores práticas de integração contínua, documentação, teste de unicidade, e controle de versão para prover um código confiável e fácil de usar. Esperamos que os aspectos teóricos e práticos apresentados aqui, sejam úteis para toda a comunidade de geocientistas.
