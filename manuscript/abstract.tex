\begin{foreignabstract}

Since the second half of the nineteenth century, a vast literature has been published on the magnetic modeling of uniformly magnetized ellipsoids. In this work, we present a integrated review about magnetic modeling of triaxial, prolate and oblate ellipsoids, with arbitrary orientation, with or without remanent magnetization and with both isotropic and anisotropic susceptibilities. We also bring a theoretical discussion regarding the commom value of isotropic susceptibility (0.1 SI), widely used by geoscientific community as the limit of which the self-demagnetization can be overlooked. Apparently this value was obtained empirically and we propose an alternative way of determining its limit, based on previous knowledge of the shape and the maximum relative error allowed in the resultant magnetization. Jointly, we provide a set of routines capable of modeling the magnetic field produce by triaxial, prolate and oblate ellipsoidal bodies. These routines are written in Python language as part of the \textit{Fatiando a Terra} package, to obtain the three magnetic field components and also the total-field anomaly for \textit{n}-sources. Examples in this work show the friendly and easy usage of the program. Hence, we hope that this work can be useful both as educational tool (e.g. Potential Methods and rock magnetism) as to applied geophysics (e.g. high susceptibility bodies characterization).

\end{foreignabstract}