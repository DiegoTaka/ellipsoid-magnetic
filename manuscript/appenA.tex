\chapter{Relação entre as derivadas das funções $f(\mathbf{r})$ e $\tilde{f}(\tilde{\mathbf{r}})$}

Sendo $\tilde{f}(\tilde{\mathbf{r}})$ a função escalar obtida transformando
$f(\mathbf{r})$ (Eq. \ref{eq:f}) do sistema de coordenas principal para um 
sistema de coordenadas local, por conveniência, reescreveremos a Eq. 
\ref{eq:coord_transformation} como:
\begin{equation}
\tilde{r}_{k} = v_{k1} \, r_{1} + v_{k2} \, r_{2} + v_{k3} \, r_{3} + c_{k} \: ,
\label{eq:r-tilde-k}
\end{equation}
em que $\tilde{r}_{k}$, $k = 1, 2, 3$, são os elementos do vetor posição
transformado $\tilde{\mathbf{r}}$ (Eq. \ref{eq:coord_transformation}),
$r_{j}$, $j = 1, 2, 3$, são os elementos do vetor posição
$\mathbf{r}$ (Eq. \ref{eq:ellipsoid_surface}),
$v_{kj}$, $j = 1, 2, 3$, são os elementos da matriz
$\mathbf{V}$ (Eq. \ref{eq:V_triaxial_prolate}),
e $c_{k}$ é uma constante definida pelas coordenadas
$x_{c}$, $y_{c}$, e $z_{c}$ do centro do corpo elipsoidal.

Considerando as funções $f(\mathbf{r})$ 
(Eq. \ref{eq:f}) e $\tilde{f}(\tilde{\mathbf{r}})$
calculadas no mesmo ponto, mas em diferentes sistemas
de coordenas, temos:
\begin{equation*}
\frac{\partial f(\mathbf{r})}{\partial r_{j}} = 
\frac{\partial \tilde{f}(\tilde{\mathbf{r}})}{\partial \tilde{r}_{1}} \,
\frac{\partial \tilde{r}_{1}}{\partial r_{j}} +
\frac{\partial \tilde{f}(\tilde{\mathbf{r}})}{\partial \tilde{r}_{2}} \,
\frac{\partial \tilde{r}_{2}}{\partial r_{j}} +
\frac{\partial \tilde{f}(\tilde{\mathbf{r}})}{\partial \tilde{r}_{3}} \,
\frac{\partial \tilde{r}_{3}}{\partial r_{j}} \: ,
\quad j = 1, 2, 3 \: ,
\end{equation*}
que, da Eq. \ref{eq:r-tilde-k}, pode ser dada por:
\begin{equation}
\frac{\partial f(\mathbf{r})}{\partial r_{j}} = 
v_{j1} \, \frac{\partial \tilde{f}(\tilde{\mathbf{r}})}{\partial \tilde{r}_{1}} +
v_{j2} \, \frac{\partial \tilde{f}(\tilde{\mathbf{r}})}{\partial \tilde{r}_{2}} +
v_{j3} \, \frac{\partial \tilde{f}(\tilde{\mathbf{r}})}{\partial \tilde{r}_{3}} \: ,
\quad j = 1, 2, 3 \: .
\label{eq:df_drj}
\end{equation}

Derivando $\frac{\partial f(\mathbf{r})}{\partial r_{j}}$
(Eq. \ref{eq:df_drj}) com respeito ao $i$-ésimo elemento 
$r_{i}$ do vetor posição $\mathbf{r}$ (Eq. \ref{eq:ellipsoid_surface}),
obtemos:
\begin{equation}
\begin{split}
\frac{\partial^{2} f(\mathbf{r})}{\partial r_{i} \, \partial r_{j}} &=
v_{j1} \, \frac{\partial}{\partial r_{i}} 
\left( \frac{\partial \tilde{f}(\tilde{\mathbf{r}})}{\partial \tilde{r}_{1}} \right) +
v_{j2} \, \frac{\partial}{\partial r_{i}} 
\left( \frac{\partial \tilde{f}(\tilde{\mathbf{r}})}{\partial \tilde{r}_{2}} \right) +
v_{j3} \, \frac{\partial}{\partial r_{i}} 
\left( \frac{\partial \tilde{f}(\tilde{\mathbf{r}})}{\partial \tilde{r}_{3}} \right) \\
&= v_{j1} \, \left( 
\frac{\partial^{2} \tilde{f}(\tilde{\mathbf{r}})}
{\partial \tilde{r}_{1} \, \partial \tilde{r}_{1}} \, v_{i1} + 
\frac{\partial^{2} \tilde{f}(\tilde{\mathbf{r}})}
{\partial \tilde{r}_{2} \, \partial \tilde{r}_{1}} \, v_{i2} + 
\frac{\partial^{2} \tilde{f}(\tilde{\mathbf{r}})}
{\partial \tilde{r}_{3} \, \partial \tilde{r}_{1}} \, v_{i3} 
\right) + \\
&+ v_{j2} \, \left( 
\frac{\partial^{2} \tilde{f}(\tilde{\mathbf{r}})}
{\partial \tilde{r}_{1} \, \partial \tilde{r}_{2}} \, v_{i1} + 
\frac{\partial^{2} \tilde{f}(\tilde{\mathbf{r}})}
{\partial \tilde{r}_{2} \, \partial \tilde{r}_{2}} \, v_{i2} + 
\frac{\partial^{2} \tilde{f}(\tilde{\mathbf{r}})}
{\partial \tilde{r}_{3} \, \partial \tilde{r}_{2}} \, v_{i3} 
\right) + \\
&+ v_{j3} \, \left( 
\frac{\partial^{2} \tilde{f}(\tilde{\mathbf{r}})}
{\partial \tilde{r}_{1} \, \partial \tilde{r}_{3}} \, v_{i1} + 
\frac{\partial^{2} \tilde{f}(\tilde{\mathbf{r}})}
{\partial \tilde{r}_{2} \, \partial \tilde{r}_{3}} \, v_{i2} + 
\frac{\partial^{2} \tilde{f}(\tilde{\mathbf{r}})}
{\partial \tilde{r}_{3} \, \partial \tilde{r}_{3}} \, v_{i3} 
\right) \\
&= \left[ \begin{array}{ccc}
v_{j1} & v_{j2} & v_{j3}
\end{array} \right] \tilde{\mathbf{F}}(\tilde{\mathbf{r}})
\left[ \begin{array}{c}
v_{i1} \\ v_{i2} \\ v_{i3}
\end{array} \right]
\end{split} \: ,
\label{eq:d2f-dridrj}
\end{equation}
em que $\tilde{\mathbf{F}}(\tilde{\mathbf{r}})$ é uma matriz $3 \times 3$, cujo $ij$-ésimo elemento é 
$\frac{\partial^{2} \tilde{f}(\tilde{\mathbf{r}})}
{\partial \tilde{r}_{i} \, \partial \tilde{r}_{j}}$.
Da Eq. \ref{eq:d2f-dridrj}, obtemos:
\begin{equation}
\mathbf{F}(\mathbf{r}) = \mathbf{V} \, 
\tilde{\mathbf{F}}(\tilde{\mathbf{r}}) \, \mathbf{V}^{\top} \: ,
\label{eq:F-V-F-tilde-VT}
\end{equation}
em que $\mathbf{F}(\mathbf{r})$ é uma matriz $3 \times 3$, cujo $ij$-ésimo elemento é 
$\frac{\partial^{2} f(\mathbf{r})}
{\partial r_{i} \, \partial r_{j}}$ e
$\mathbf{V}$ é definido pelas Eq. 
\ref{eq:V_triaxial_prolate}, dependendo do tipo de
elipsoide. Como pode ser notado, as matrizes $\mathbf{F}(\mathbf{r})$ e
$\tilde{\mathbf{F}}(\tilde{\mathbf{r}})$ representam as Hessianas
das funções $f(\mathbf{r})$ (Eq. \ref{eq:f})
e $\tilde{f}(\tilde{\mathbf{r}})$, respectivamente.
Além disso, o tensor de depolarização $\mathbf{N}(\mathbf{r})$
(Eq. \ref{eq:H-M-uniform}) cujo ij-ésimo elemento é dado pela Eq. \ref{eq:nij} pode se 
reescrito usando a matriz $\mathbf{F}(\mathbf{r})$
como:
\begin{equation}
\mathbf{N}(\mathbf{r}) = - \frac{1}{4 \pi} \mathbf{F}(\mathbf{r}) \: .
\label{eq:N-F}
\end{equation}
Usando apropriadamente a ortogonalidade de matrizes
$\mathbf{V}$,podemos reescrever a Eq. \ref{eq:F-V-F-tilde-VT}
como:
\begin{equation}
\tilde{\mathbf{F}}(\tilde{\mathbf{r}}) = \mathbf{V}^{\top} \, 
\mathbf{F}(\mathbf{r}) \, \mathbf{V} \: .
\label{eq:F-tilde-VT-F-V}
\end{equation}
Finalmente, multiplicando os dois lados da Eq. \ref{eq:F-tilde-VT-F-V}
por $-\frac{1}{4 \pi}$ e usando a Eq. \ref{eq:N-F},
concluímos que 
\begin{equation}
\tilde{\mathbf{N}}(\tilde{\mathbf{r}}) = 
\mathbf{V}^{\top} \, \mathbf{N}(\mathbf{r}) \, \mathbf{V} \: .
\label{eq:N-tilde-VT-N-V}
\end{equation}