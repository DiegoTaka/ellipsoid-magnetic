\begin{abstract}

Desde a segunda metade do século dezenove, uma vasta literatura sobre a modelagem magnética de corpos elipsoidais foi publicada.
Apresenta-se, nesta dissertação, uma revisão integrada da teoria sobre a modelagem magnética de elipsoides triaxiais, prolatos e oblatos, com orientações arbitrárias, com ou sem magnetização remanente e com suscetibilidade magnética tanto isotrópica como anisotrópica. Levantamos também uma discussão teórica acerca do valor comumente encontrado na literatura de 0,1 SI para a susceptibilidade isotrópica, cuja desmagnetização deve ser levada em consideração. Este valor parece ter sido obtido de forma empírica e tem sido usado em larga escala pela comunidade geocientífica. Neste trabalho propomos uma definição deste valor limite, com base no conhecimento prévio do intérprete sobre a forma do corpo e sobre o erro relativo máximo permitido na magnetização resultante calculada na modelagem. Conjuntamente, apresenta-se um conjunto de rotinas capazes de realizar a modelagem direta do campo magnético gerado por fontes elipsoidais triaxiais, prolatas e oblatas. Os dados sintéticos foram produzidos com uma implementação utilizando linguagem Python, com base no pacote \textit{Fatiando a Terra} para obtenção das três componentes de campo magnético e também da anomalia de campo total para \textit{n}-fontes. Exemplos realizados na dissertação demonstram a facilidade de uso do programa.
Espera-se que isto torne o trabalho relevante tanto como ferramenta educacional
(e.g., métodos potenciais e magnetismo de rochas) como para a geofísica aplicada (e.g., caracterização de corpos mineralizados de alta susceptibilidade).

\end{abstract}