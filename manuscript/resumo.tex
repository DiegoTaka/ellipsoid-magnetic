\begin{abstract}

Desde a segunda metade do século dezenove, uma vasta literatura sobre a modelagem magnética de corpos elipsoidais foi publicada.
Apresenta-se, nesta dissertação, uma revisão integrada da teoria sobre a modelagem magnética de elipsoides triaxiais, prolatos e oblatos, com orientações arbitrárias, com ou sem magnetização remanente e com suscetibilidade magnética tanto isotrópica como anisotrópica. Este trabalho também apresenta uma discussão teórica acerca do valor de 0,1 SI que é comumente utilizado na literatura como a susceptibilidade isotrópica, a partir do qual desmagnetização deve ser levada em consideração na modelagem. Este valor parece ter sido obtido de forma empírica e pouco tem sido discutido sobre isso na comunidade geocientífica. Este trabalho propõe uma definição deste valor limite com base no conhecimento prévio do intérprete sobre a forma do corpo e sobre o erro relativo máximo permitido na magnetização resultante calculada na modelagem. Conjuntamente, apresenta-se um conjunto de rotinas capazes de calcular o campo magnético gerado por fontes elipsoidais triaxiais, prolatas e oblatas. As rotinas foram desenvolvidas em linguagem \textit{Python}, com base no pacote \textit{Fatiando a Terra}.
Exemplos apresentados nesta na dissertação demonstram a facilidade de uso destas rotinas.
Estas rotinas podem ser utilizadas tanto como ferramenta educacional
(e.g., métodos potenciais e magnetismo de rochas) como para a geofísica aplicada (e.g., caracterização de corpos mineralizados de alta susceptibilidade) e estão disponibilizadas livremente no link https://github.com/DiegoTaka/ellipsoid-magnetic para toda a comunidade científica.

\end{abstract}