\begin{abstract}

Apresenta-se, nesta dissertação, a modelagem direta do
campo magnético gerado por fontes elipsoidais triaxiais,
prolatas e oblatas.
Os dados sintéticos foram produzidos com uma
implementação utilizando linguagem Python, para obtenção 
dos mapas das três componentes de campo magnético e também
da anomalia total de campo aproximada para n-fontes.
Como o modelo elipsoidal permite uma grande
flexibilidade em sua forma geométrica, apresentaremos
os resultados obtidos para validação do código comparando 
com outras formas, como esferas e também por simulações numéricas que
incluem o efeito de anisotropia e o de desmagnetização 
no cálculo do vetor de magnetização resultante da fonte, uma
vez que esta pode ser calculada de forma analítica em elipsoides.
Isto torna o trabalho relevante tanto como ferramenta educacional
(e.g., métodos potenciais e magnetismo de rochas) e geofísica
aplicada (e.g., caracterização de corpos mineralizados de alta 
susceptibilidade).
O código está disponibilizado livremente para uso da comunidade
científica pelo software \textit{Fatiando a Terra}.

\end{abstract}